\documentclass[dvipdfmx,11pt,notheorems]{beamer}
%%%% 和文用 %%%%%
\usepackage{bxdpx-beamer}
\usepackage{pxjahyper}
\usepackage{minijs}%和文用
\renewcommand{\kanjifamilydefault}{\gtdefault}%和文用

%%%% スライドの見た目 %%%%%
\usetheme{Madrid}
\usefonttheme{professionalfonts}
\setbeamertemplate{frametitle}[default][center]
\setbeamertemplate{navigation symbols}{}
\setbeamercovered{transparent}%好みに応じてどうぞ)
\setbeamertemplate{footline}[page number]
\setbeamerfont{footline}{size=\normalsize,series=\bfseries}
\setbeamercolor{footline}{fg=black,bg=black}
%%%%

%%%% 定義環境 %%%%%
\usepackage{amsmath,amssymb}
\usepackage{amsthm}
\theoremstyle{definition}
\newtheorem{theorem}{定理}
\newtheorem{definition}{定義}
\newtheorem{proposition}{命題}
\newtheorem{lemma}{補題}
\newtheorem{corollary}{系}
\newtheorem{conjecture}{予想}
\newtheorem*{remark}{Remark}
\renewcommand{\proofname}{}
%%%%%%%%%

%%%%% フォント基本設定 %%%%%
\usepackage[T1]{fontenc}%8bit フォント
\usepackage{textcomp}%欧文フォントの追加
\usepackage[utf8]{inputenc}%文字コードをUTF-8
\usepackage{otf}%otfパッケージ
\usepackage{lxfonts}%数式・英文ローマン体を Lxfont にする
\usepackage{bm}%数式太字
%%%%%%%%%%
 
\title[略タイトル]{\LaTeX +Beamer でスライドを作ろう!}%[略タイトル]{タイトル}
\author[Xaro]{Xaro Cydeykn}%[略名前]{名前}
\institute[JPN]{Tokyo, Japan}%[略所属]{所属}
\date{\today}%日付
\begin{document}

\begin{frame}[plain]\frametitle{}
\titlepage %表紙
\end{frame}

\begin{frame}\frametitle{Contents}
\tableofcontents %目次
\end{frame}

\section{何故Beamerを使うのか?}
\begin{frame}\frametitle{PowerPointじゃだめなの?}
\begin{alertblock}{PowerPointの問題点}
\begin{itemize}
\item PowerPointのバージョン・OSに依存する 
\item PowerPointは有料
\item 数式が(相当がんばらないと)汚い 
\end{itemize}
\end{alertblock}
\end{frame}

\begin{frame}\frametitle{何故Beamerを使うのか?}
\begin{block}{Beamerのよいところ}
\begin{itemize}
\item 論文・レジメの再利用が容易 
\item レイアウトの微調整は\LaTeX に任せることができる 
\item 数式がきれい 
\end{itemize}
\end{block}

\begin{exampleblock}{数式の例} 
\begin{equation*}
\frac{\pi}{2} =\left( \int_{0}^{\infty} \frac{\sin x}{\sqrt{x}} dx \right)^2 =\sum_{k=0}^{\infty} \frac{(2k)!}{2^{2k}(k!)^2} \frac{1}{2k+1} 
=\prod_{k=1}^{\infty} \frac{4k^2}{4k^2 - 1}
\end{equation*}
\end{exampleblock}
\end{frame}


\section{具体例}

\begin{frame}\frametitle{定理環境の例}
\begin{theorem}[Fermat]
$a^{p-1} \equiv 1 \pmod{p}$
\end{theorem}
\pause
\begin{theorem}[Wilson]
\begin{equation}
(p-1)! \equiv 1 \pmod{p}
\end{equation}
\end{theorem}
\end{frame}

\begin{frame}<1-2>\frametitle{オーバーレイ}
\onslide*<1>{
\Large{これは1枚目です}
}
\onslide*<2>{
これは2枚目です
\begin{theorem}[Euclid]
There is no largest prime number.
\end{theorem}
}
\end{frame}

\begin{frame}\frametitle{色もつけれるよ}
  {\color{red} red}(\alert{alert}),
  {\color{blue} blue}(\structure{structure}),
  {\color{green} green},
  {\color{cyan} cyan},
  {\color{magenta} magenta},
  {\color{yellow} yellow},
  {\color{black} black},
  {\color{darkgray} darkgray},
  {\color{gray} gray},
  {\color{lightgray} lightgray},
  {\color{orange} orange},
  {\color{violet} violet},
  {\color{purple} purple},
  {\color{brown} brown},
\end{frame}

\begin{frame}\frametitle{いろんなブロック}
\begin{block}{ブロック}
これは普通のブロックです
\end{block}

\begin{alertblock}{警告ブロック}
警告!これは警告ブロックだ!
\end{alertblock}

\begin{exampleblock}{例ブロック}
例えば、こんなブロックです。
\end{exampleblock}
\end{frame}

\begin{frame}<1-2>\frametitle{画像も貼れるよ}
\onslide*<1>{
このように画像を貼れるよ
%\begin{figure}[htb]
%\centering
%\includegraphics[width=12cm,clip]{dummygraph.pdf}
%\caption{$f(x)=e^{-\frac{x}{10}}\sin(x)$}
%\end{figure}%
}
\onslide*<2>{
画像や表は各自用意してね
%\begin{figure}[htb]
%\centering
%\includegraphics[width=8cm,clip]{sym4.pdf}
%\caption{Cayley graph of $\mathfrak{S}_{4}$}
%\end{figure}%
}
\end{frame}

\begin{frame}\frametitle{まとめ}
\LARGE{大事なのは中身です!}
\end{frame}

\begin{frame}\frametitle{}
{\Large ありがとうございました}
\end{frame}
\appendix

\newcounter{finalframe}
\setcounter{finalframe}{\value{framenumber}}

\begin{frame}[containsverbatim]\frametitle{dvipngの使い方(1)}
\begin{block}{この様なファイルを用意する}
\tiny{
\begin{verbatim*}
\documentclass[43pt]{jsarticle}
\usepackage{amsmath}
\usepackage{lmodern}
\pagestyle{empty}
\begin{document}
\begin{equation*}
\sum_{k=0}^{\infty} \frac{(2k)!}{2^{2k}(k!)^2} \frac{1}{2k+1}=\frac{\pi}{2} 
\end{equation*}
\end{document}
\end{verbatim*}
}
\end{block}
\end{frame}

\begin{frame}[containsverbatim]\frametitle{dvipngの使い方(2)}
\begin{block}{使い方(コマンドライン)}
\scriptsize{
\begin{verbatim*}
latex dvipng-sample.tex
dvipng dvipng-sample.dvi -T tight -bd 1000
\end{verbatim*}
}
\end{block}
\end{frame}

\setcounter{framenumber}{\value{finalframe}}
\end{document}
